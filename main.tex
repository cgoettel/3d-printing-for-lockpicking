\documentclass{acm_proc_article-sp}

\title{Typesetting Biblical Hebrew Poetry}
\author{Colby Goettel}

\usepackage{mathtools}
% Hebrew
\usepackage{xltxtra}
\newfontfamily{\sblh}[Script=Hebrew]{SBL Hebrew}
\newcommand\Hebrew[1]{{\large\sblh #1}}

\begin{document}
\maketitle

changed verse numbers to superscript to get them out of the way
    red so they don't look like footnotes
added commands to handle formatting the Hebrew
    need to consolidate these (but not for this paper)

NOTES FROM ISSUE #1:
Mimic layout of verses in the BHS. Possibly using this: http://mirror.math.ku.edu/tex-archive/macros/latex/contrib/verse/verse.pdf
    This ended up not being used. It was simpler for this complex purpose to build something unique.

Need to preserve indentation (both at beginning of line and in middle).

**Specifications:**
- [x] Each line is indented at least once.
    using minipages did this naturally.
    tried using \hbox and \pbox, but neither did what needed to be done. Didn't want a box around it and the indentation didn't work when a new-line occurred in the middle of a verse.
- [x] Not every verse starts on a new line. Verses can start in the middle of an existing line.
    had to create the \vn macro to handle this.
- [x] Each line is comprised of sections:
  1. Each section is separated by a space.
    played around with different spacings, purely aesthetic, and ended up using the 1em spacing. Anything else threw the word spacing off by too much.
  2. There should be no trailing whitespace.
    so you can't just constantly throw in the 1em space. It needed to be precise.
  3. Can't assume that there will always be one or two sections, sometimes three or more appear. It is always variable.
    Currently handling from aa to dd.

**Previous notes:**
- [x] Switch out of two column view
- [ ] ~~Possibly: use a new command \tab{} which would insert the spacing (probably not the best idea)~~
- [ ] ~~Use the \itemcountmarker for verses~~
- [x] If a verse occurs mid-line, put the number in square brackets.
- [ ] Don't include verse numbers as part of the indentation (do it like the BHS does).

I've been translating Hebrew for four years now and have fallen in love with the format of Biblical Hebrew poetry.

\begin{align*}
    \text{\Hebrew{מֵעֲוֹנֹתֵ֑ינוּ מְדֻכָּ֖א~~~~ מִפְּשָׁעֵ֔נוּ מְחֹלָ֣ל וְהוּא֙}}&\,^5 \\
    \text{\Hebrew{נִרְפָּא־לָֽנוּ׃ וּבַחֲבֻרָתֹ֖ו~~~~ עָלָ֔יו שְׁלֹומֵ֙נוּ֙ מוּסַ֤ר}}&
\end{align*}
% \begin{align}
%     &^5\,\text{He was wounded for our transgressions, He was bruised for our iniquities. Upon Him was the} \\
%     &\text{punishment through which we receive peace. We are healed by His stripes.}
% \end{align}
% \begin{align}
%     ^5\,&\text{He was wounded for our transgressions, ~~~~He was bruised for our iniquities.} \\
%         &\text{Upon Him was the punishment through which we receive peace. ~~~~We are healed by His stripes.}
% \end{align}

\bibliographystyle{abbrv}
\bibliography{references}
\nocite{*}

\end{document}
